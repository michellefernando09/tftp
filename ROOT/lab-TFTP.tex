%%%%%%%%%%%%%%%%%%%%%%%%%%%%%%%%%%%%%%%%%%%%%%%%%%%%%%%%%%%%%%%%%%%%%%%%%%%
% FILE    : lab-TFTP.tex
% AUTHOR  : (C) Copyright 2015 by Peter C. Chapin
% SUBJECT : Trivial FTP
%
% In this lab the students will write a peer to peer file transfer utility using TFTP. In the
% process they will learn something about how transport protocols work... and what they have to
% think about.
%%%%%%%%%%%%%%%%%%%%%%%%%%%%%%%%%%%%%%%%%%%%%%%%%%%%%%%%%%%%%%%%%%%%%%%%%%%

% ++++++++++++++++++++++++++++++++
% Preamble and global declarations
% ++++++++++++++++++++++++++++++++
\documentclass[twocolumn]{article}

%\pagestyle{headings}
%\setlength{\parindent}{0em}
%\setlength{\parskip}{1.75ex plus0.5ex minus0.5ex}

% +++++++++++++++++++
% The document itself
% +++++++++++++++++++
\begin{document}

% ----------------------
% Title page information
% ----------------------
\title{Trivial FTP}
\author{Peter C. Chapin\thanks{PChapin@vtc.vsc.edu}\\
  Vermont Technical College}
\date{Last Revised: February 5, 2015}
\maketitle

\section*{Introduction}

In this lab you will write programs that use the TFTP protocol to transfer files from one
machine to another. The intent is for you to make a utility that could actually be useful to
you. In the process you will explore some of the issues that transport protocols in general have
to worry about. TFTP is built on top of UDP and it solves many of the issues that TCP solves...
although in a much simpler and more limited way.

TFTP is described by RFC-1350. Please read that RFC before doing this lab (it's about 24 KiB).
Also review RFC-2347, RFC-2348, and RFC-2349 (all are short). These later RFCs describe
extensions to the TFTP protocol that allow the client and server to exchange options such as
block size and time-out intervals. I may ask you to implement some or all of these options as
well (depending on how the time goes).

\section{The Client}

The client should accept the \emph{name} of the server host and the port number (if not the
default of 69) on the command line. The client should then go into a loop where it prompts the
user to enter a command. If the user enters the name of a file as the ``command,'' your client
should contact the server and get that file. The client should provide the user with some
feedback about the transfer such as the number of blocks transfered. It would also be nice to
time the transfer and display how much time was needed and the effective data transfer rate.
When the transfer is complete the client should prompt the user again. If the user types
``!quit'' at the prompt the client should end.

\section{Example Session}

Start the client like this

\begin{verbatim}
$ client localhost 9000
\end{verbatim}

Here I'm assuming the client is named \texttt{client} and that the server is running on the
non-default port of 9000. The client prints a prompt and allows the user to get multiple files,
one at a time. For example,

\begin{verbatim}
get> somefile.txt
###
1234 bytes in 0.32s (3856 bytes/s)
get> dir/somefile.txt
###########
5678 bytes in 22.45s (253 bytes/s)
get> !quit
\end{verbatim}

All the file names should be relative to the working directory at the server side. However,
regardless of their location on the server, the files should be stored in the current directory
on the client side. Thus the file \texttt{dir/somefile.txt} would be stored as just
\texttt{somefile.txt} on the client.

The example feedback provided by the client above is to print a `\#' mark for each block
received followed by the exact size of the transfered file at the end of the transfer. You can
provide somewhat different feedback if you like, but you do need to let the user know that the
program is actually doing something.

The user ends the session by typing ``!quit''. The client is only required to get files from the
server. No support for sending files is required although it could be an interesting extension.
Only the octet transfer mode needs to be supported.

\section{The Server}

The server should use the default TFTP port or, if an overriding port is provided on the command
line, it should use that port instead. It should log all activity to a log file, the precise
nature of which is unspecified here. However, it might be nice to use the regular system logging
facility. In any case the server should have no other user interface (except perhaps debugging
messages).

The server should use the current working directory as the root directory from which files are
served. A nice extension might be to have the server read a configuration file where, perhaps,
multiple directories could be specified, but that is not required here.

The server should handle multiple, concurrent clients and properly demonize itself when it
starts.

\textit{Security Notes}. The server should ``cleanse'' the path of requested files so that
clients can't access unauthorized files by creatively using the ``..'' path component. The
server should also only serve files which have public read access and should only allow existing
files with public write access to be overwritten. In the case of file write requests the server
should impose a limit (not specified here) on the largest size of an accepted file. The server
should not allow reads or writes of any file with a name starting with `.' or that contains
``unusual'' characters.

The ability to accept write requests is not required in a first version of the server but could
be considered a reasonable extension. Only the octet transfer mode needs to be specified.

\section{Testing}

There are many issues to consider here: unit testing, ``live'' debugging messages, etc, etc.
However, I want to point out in this section that you can test each program independently
against the standard TFTP client and server that come with the system. You may have to install
appropriate TFTP packages to obtain those programs.

\section{Report}

Write a full report for this lab. Include commented listings of your programs (or at least of
the interesting parts of your programs) as well as some verbal discussion about how your
programs work and about some of the issues that arise in systems of this nature. For example,
consider the effect of various kinds of network errors on your programs (lost packets,
duplicated packets, or packets arriving out of order). Consider the effect of using a longer (or
shorter) timeout interval. Would there be any advantage to using a larger block size? These are
some of the points that would be worth discussing in your report. There may be others as well.

\end{document}
